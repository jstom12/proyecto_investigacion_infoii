% 		INSTRUCTIVO PARA CONSTRUIR EL ENSAYO DE FICCIÓN EN LATEX
%					v.1.3
%				elaborado por @rmuriel
%		---------------------------------------------------
		
% Este instructivo se crea para facilitar la comprensión sobre la estructura de un trabajo escrito en LaTeX, y de paso para que se comprenda una forma (entre las miles que existen) en que se podría construir un «ensayo de ficción».

% Antes de empezar, para algunos será útil que lo diga, se observará que el documento está dividido en títulos en mayúscula y comentarios en color azul (como este). Corresponden a las partes habituales de un documento LaTeX y mis explicaciones. La parte que van a modificar se llama «CUERPO DEL DOCUMENTO». No tienen que saber código LaTeX para hacer este trabajo, sólo se dejan guiar por esta plantilla y listo. Más allá de leer juiciosamente este instructivo, no necesitan más!! :)

% Lo que esté en latín y en color negro es lo que ustedes van a modificar. Verán que es muy fácil... abran la mente y déjense guiar! :)

% Comencemos!! 

%================================================================»
% 0 - RAZONES POR LAS QUE HACEMOS ESTE TRABAJO EN LATEX
%================================================================»

% 1. LaTeX es el lenguaje del mundo académico. Las mejores universidades del mundo funcionan así, en todas existe al menos una cátedra permanente de LaTeX. En Colombia, lastimosamente, sólo lo usa la Universidad Nacional, la Universidad de Antioquia y la Universidad de los Andes. Cada una de estas universidades tiene formatos (plantillas LaTeX) para trabajos de semestre, tesis, informes de laboratorio, ensayos y exámenes parciales. 

% 2. En dos aspectos importantes es el procesador de texto más efectivo que existe: rendimiento de la máquina (no hace que se cuelgue) y acabado editorial del documento (La forma final del documento es profesional y responde a los cánones editoriales internacionales).

% 3. Es software libre y multiplataforma. Se puede trabajar desde casi cualquier sistema operativo respetable. No necesita ser pirateado y está disponible para su descarga las 24 horas del día. Sin contar con que existe https://www.writelatex.com, que hace el trabajo mucho más cómodo de manera online. 

% 4. Automatiza procedimientos mecánicos típicos en la construcción de documentos:  autonumeración de fórmulas, generación de listas, creación de índices de contenido, de tablas, figuras y terminológicos, etc. 

% 5. La preocupación por la forma se la deja uno al computador. Uno no tiene que pensar en las márgenes, en las negritas de los títulos, en el tipo de letra, etc... De esas formalidades se encarga LaTeX... Uno sólo se concentra en producir contenido, que es para eso que se inventaron los procesadores de texto en una computadora. Preocuparse por la forma es como si no se hubiera superado la época de la máquina de escribir. 

% 6. Permite el uso de bases de datos bibliográficas con BibTeX. Se ahorra tiempo a la hora de citar textos y hacer listados de publicaciones. Basta con hacer una vez la base bibliográfica y uno sólo debe «llamar» las referencias usadas para cualquier cantidad de textos que uno escriba. En esto LaTeX está conectado con Mendeley, la base de datos bibliográfica más importante de la actualidad en el mundo académico. 

% y como si esto fuera poco...

% 7. Con LaTeX se permiten hacer comentarios en cualquier sección del texto sin que aparezcan en el documento final. Basta con introducir un signo de porcentaje (%) antes de empezar a escribirlos. 

% De modo que... empecemos desde ya a usar LaTeX!!!!

%================================================================»
% I - PREÁMBULO
%================================================================»

% Antes de escribir el texto como tal, para LaTeX es importante clarificar algunos aspectos básicos sobre la naturaleza del documento que se va a escribir. Esta primera parte se llama «PREÁMBULO» y es el lugar donde se clarifican los siguientes aspectos:

% - Tamaño de hoja y de fuente
% - Tipo de documento: libro, artículo, informe, etc.
% - Paquetes de información: español, márgenes, copiado pdf, colores, gráficos, etc. 
% - Autor, título y fecha
% - Algunos ajustes a la estética del documento general

%----------------------------------------------------------------»
% a - Definición de la Clase del Documento 
%----------------------------------------------------------------»

\documentclass[12pt,letterpaper]{article}

%----------------------------------------------------------------»
% b - Paquetes para trabajar en español 
%----------------------------------------------------------------»

\usepackage[utf8]{inputenc}
\usepackage[spanish]{babel}

%----------------------------------------------------------------»
% c - Paquetes para solucionar el copiado del pdf
%----------------------------------------------------------------»

\usepackage{times}		
\usepackage[T1]{fontenc}	

%----------------------------------------------------------------»
% d - Paquetes especiales (Según las necesidades del documento)
%----------------------------------------------------------------»
\usepackage{natbib}
\usepackage[colorinlistoftodos]{todonotes} %Para insertar notas al lado
\usepackage{graphicx} %Para usar imágenes
\usepackage{tikz} %Para construir gráficos con código
\usepackage{epigraph} %Hacer epígrafes
\usepackage{multicol} %Construir múltiples columnas en el documento
\usepackage{color} %Para darle color a la fuentes
\usepackage{soul} %Para tachar palabras
\usepackage{ulem} %Para subrayados y tachados especiales (\uuline, \uwave, \xout) Aunque casi nunca se usan, a veces pueden introducirse para remarcar algo. 

%----------------------------------------------------------------»
% e - Paquete para generar links (Si el doc. tiene hipervínculos)
%----------------------------------------------------------------»

\usepackage[backref]{hyperref}	% Soporte para generación de Links - Ojalá siempre el último paquete nombrado
\hypersetup{pdfborder={0 0 0}}	% Quitarle los bordes a los links

%----------------------------------------------------------------»
% f - Arreglos sobre la estética de los párrafos (Opcional)
%----------------------------------------------------------------»

\setlength\parindent{0pt}	% Si se quiere suprimir la sangría de los párrafos
\setlength{\parskip}{2mm}	% Si se quiere espaciar todos los párrafos

%----------------------------------------------------------------»
% g - Autor, título y fecha del Documento
%----------------------------------------------------------------»

\author{JUAN MANUEL RESTREPO GALARCIO\thanks{INGENIERIA DE TELECOMUNICACIONES, UNIVERSIDAD DE ANTIOQUIA, 2020}}
\title{HILOS EN LOS MICROPROCESADORES.}
\date{\today} 

%================================================================»
% II - CUERPO DEL DOCUMENTO
%================================================================»

% Después de todo el preámbulo nos adentramos en la escritura del trabajo. El CUERPO DEL DOCUMENTO en LaTeX siempre inicia con las siguientes dos instrucciones:

\begin{document}
\maketitle

% En el CUERPO DEL DOCUMENTO es donde vamos a encontrar:

% - Abstract
% - Secciones y subsecciones
% - Tabla de contenido
% - Tablas
% - Gráficos
% - Notas al pie y al márgen
% - Párrafos especiales (cita)
% - Bibliografía

%----------------------------------------------------------------»
% a - Creación del resumen (Abstract)
%----------------------------------------------------------------»

% El abstract es el resumen del ensayo. Se expone, entre cuatro y siete líneas, la naturaleza del escrito, su tema, el tipo de indagación y los intereses del texto. 

\section{INTRODUCCIÓN:}

En este escrito se busca realizar un informe respecto a los hilos de procesamiento y resolver: ¿que es un hilo de procesamiento? ¿como se comportan los hilos de procesamiento en el hardware? ¿como se comportan los hilos de procesamiento en el software?



\section{¿QUE ES UN HILO?:}

Un hilo de procesamiento, o conocido también subproceso, es una función con la cual se puede realizar tareas simultaneas dentro del procesador (CPU); Recordemos que los CPU funcionan de manera secuencial pero los mas actuales tiene una arquitectura que les permite ejecutar múltiples tareas al mismo tiempo, esto gracias a que los CPU actuales poseen varios núcleos que son también conocidos como subprocesadores.

El funcionamiento de los núcleos consiste en recibir tareas por parte de los hilos de procesamiento y ejecutarlas; Cada núcleo puede recibir entre uno o dos hilos de procesamiento, por lo tanto, un CPU estándar de 4 núcleos puede procesar hasta 8 tareas al mismo tiempo.

Actualmente esta tecnología es muy importante porque no es común para un ordenador o un celular ejecutar solo una tarea al mismo tiempo y, de no estar incorporada esta tecnología podríamos decir que, si un procesador con 2 núcleos es el doble de eficiente que un procesador con un solo núcleo, el tiempo de ejecución se vería duplicado.

El primer procesador que poseía mas de un subprocesador y tenía la capacidad de ejecutar mas de una tarea simultáneamente fue el IBM POWER4 en el años 2001 pero no fue un procesador distribuido hacia las masas; El que marcó un antes y después en la multitarea dentro de la CPU fue la compañía Intel al sacar, en el año 2006, su gama de procesadores Dual Core aumentando con creces las velocidad de procesamiento de las CPU.

\section{HILOS EN EL HARDWARE:}

Los hilos de procesamiento funcionan por medio de los subprocesadores de la CPU; Antes de la llegada de los procesadores multinucleo Intel había logrado ejecutar dos hilos de procesamiento en un solo procesador pero, con la llegada del Dual Core, esta tecnología fue dejada de lado hasta unos años después que se empezó a implementar para hacer que 4 núcleos pudiesen ejecutar 8 hilos de procesamiento simultáneamente.

El funcionamiento de los hilos de procesamiento dentro del hardware es el siguiente: Un hilo principal divide las tareas que recibe en en varios hilos que son procesados en alguno de los núcleos disponibles; Al terminar todas las tareas, el tiempo en el que estuvo en ejecución el hilo principal corresponde al tiempo del hilo de procesamiento que tardó mas en realizar su tarea y así ya estarían otra vez disponibles los hilos para realizar mas tareas.

Un ejemplo de la implementación de hilos de procesamiento de manera eficiente podría ser un programa que simule el funcionamiento de semáforos; cada hilo tiene la tarea de cambiar el color del semáforo al mismo tiempo cosa que no sería posible realizar en los procesos secuenciales de un procesador con un solo hilo de procesamiento.


\section{HILOS EN EL SOFTWARE:}

...

\section{BIBLIOGRAFÍA:}








%----------------------------------------------------------------»
% b - Escribir el Epígrafe (Opcional)
%----------------------------------------------------------------»

% Uno puede escribir o no un epígrafe al principio de un ensayo. Ustedes quizá lo han visto con frecuencia en diferentes tipos de escritos (ensayos, novelas, etc.) - Lo importante es que el epígrafe aluda a algo importante que usted quiere comunicar en el ensayo. 

%\epigraph{El uso de los poderes y la igualdad de condiciones se ve reflejado en las nuevas generaciones}{Proverbio de la luna}

%----------------------------------------------------------------»
% c - Inicio de las secciones del documento
%----------------------------------------------------------------»


% ----------------------
% La intrucción \underline se usa para subrayar frases. 
% ----------------------
%----------------------------------------------------------------»
% c - Bibliografía
%----------------------------------------------------------------»

% El entorno «thebibliography» nos sirve para construir la bibliografía. Cada \bibtem es una referencia que hemos usado en nuestro documento. 

% Para citar las referencias usamos el comando \cite{etiqueta}, tal y como se hizo en el último párrafo de esta plantilla.  Por supuesto, la «etiqueta» es el nombre que le hemos dado a la referencia. En el caso del primer libro de esta bibliografía vemos que la etiqueta es «ejemplo», las otras son «libro1», «libro2», etc. Usted puede usar cualquier etiqueta siempre y cuando no se repita en otra referencia. Cada referencia tiene etiqueta única.



\bibliographystyle{apalike}
\bibliography{bibliografia.bib}



%================================================================»
% EXPLICACIONES FINALES
%================================================================»
%----------------------------------------------------------------»
% Signos en LaTeX
%----------------------------------------------------------------»

% Como se ha notado, escribir el signo % (porcentaje) produce «comentarios» dentro del código, explicaciones que no son tomadas en cuenta a la hora de «compilar» el código escrito. Si se quiere incorporar un signo % (porcentaje) como parte del texto que se está escribiendo debe escribirse con la barra de instrucción habitual, así: \% 

% Hay otros signos a los que también es necesario antecederlos de la barra \ - Son los siguientes:

% \		carácter inicial de comando			Se escribiría: \tt\char‘\\
% { }	abre y cierra bloque de código		Se escribiría: \{, \}
% $		abre y cierra el modo matemático		Se escribiría: \$
% &		tabulador (en tablas y matrices)		Se escribiría: \&
% #		señala parámetro en las macros		Se escribiría: \_ , \^{}
% _, ^	para subíndices y exponentes			Se escribiría: \#
% ~		para evitar cortes de renglón			Se escribiría: \~{}

%----------------------------------------------------------------»
% Cambios en la estética de las palabras
%----------------------------------------------------------------»

% Este es el listado de las instrucciones básicas:

% - Negrita: 	\textbf{}
% - Itálica: 	\textit{}
% - Slanted:		\textsl{}
% - Sans Serif:	\textsf{}
% - Versalitas:	\textsc{}
% - Typewriter: 	\texttt{}
% - Enfático:	\emph{}

% Lo que se escriba dentro de los corchetes de cada instrucción será lo que se verá modificado en el texto. Ejemplo:

% \sc{Esto es una frase en versalitas}

% Por supuesto, la anterior instrucción no compilará en este documento porque la antece un signo de % (porcentaje), que es el signo de los «comentarios». Pero, pruebe en el texto normal y verá los cambios con cada una de las anteriores instrucciones. 

%--------------------------------»»
% NOTA IMPORTANTE
% Si alguien quiere anexar tablas o gráficos al documento, le recomiendo acercarse a la sección que lo explica en los manuales, guías o instructivos que están en BlackBoard. 
%--------------------------------»»

\end{document}